\documentclass[journal,12pt,twocolumn]{IEEEtran}
%
\usepackage{setspace}
\usepackage{gensymb}
%\doublespacing
\singlespacing

%\usepackage{graphicx}
%\usepackage{amssymb}
%\usepackage{relsize}
\usepackage[cmex10]{amsmath}
%\usepackage{amsthm}
%\interdisplaylinepenalty=2500
%\savesymbol{iint}
%\usepackage{txfonts}
%\restoresymbol{TXF}{iint}
%\usepackage{wasysym}
\usepackage{amsthm}
%\usepackage{iithtlc}
\usepackage{mathrsfs}
\usepackage{txfonts}
\usepackage{stfloats}
\usepackage{bm}
\usepackage{cite}
\usepackage{cases}
\usepackage{subfig}
%\usepackage{xtab}
\usepackage{longtable}
\usepackage{multirow}
%\usepackage{algorithm}
%\usepackage{algpseudocode}
\usepackage[utf8]{inputenc}
\usepackage{tikz}
\usetikzlibrary{positioning}
\usepackage{enumitem}
\usepackage{mathtools}
\usepackage{steinmetz}
\usepackage{tikz}
\usepackage{circuitikz}
\usepackage{verbatim}
\usepackage{tfrupee}
\usepackage[breaklinks=true]{hyperref}
%\usepackage{stmaryrd}
\usepackage{tkz-euclide} % loads  TikZ and tkz-base
%\usetkzobj{all}
\usetikzlibrary{calc,math}
\usepackage{listings}
    \usepackage{color}                                            %%
    \usepackage{array}                                            %%
    \usepackage{longtable}                                        %%
    \usepackage{calc}                                             %%
    \usepackage{multirow}                                         %%
    \usepackage{hhline}                                           %%
    \usepackage{ifthen}                                           %%
  %optionally (for landscape tables embedded in another document): %%
    \usepackage{lscape}     
\usepackage{multicol}
\usepackage{chngcntr}
%\usepackage{enumerate}

%\usepackage{wasysym}
%\newcounter{MYtempeqncnt}
\DeclareMathOperator*{\Res}{Res}
%\renewcommand{\baselinestretch}{2}
\renewcommand\thesection{\arabic{section}}
\renewcommand\thesubsection{\thesection.\arabic{subsection}}
\renewcommand\thesubsubsection{\thesubsection.\arabic{subsubsection}}

\renewcommand\thesectiondis{\arabic{section}}
\renewcommand\thesubsectiondis{\thesectiondis.\arabic{subsection}}
\renewcommand\thesubsubsectiondis{\thesubsectiondis.\arabic{subsubsection}}

% correct bad hyphenation here
\hyphenation{op-tical net-works semi-conduc-tor}
\def\inputGnumericTable{}                                 %%

\lstset{
%language=C,
frame=single, 
breaklines=true,
columns=fullflexible
}
%\lstset{
%language=tex,
%frame=single, 
%breaklines=true
%}

\begin{document}
%


\newtheorem{theorem}{Theorem}[section]
\newtheorem{problem}{Problem}
\newtheorem{proposition}{Proposition}[section]
\newtheorem{lemma}{Lemma}[section]
\newtheorem{corollary}[theorem]{Corollary}
\newtheorem{example}{Example}[section]
\newtheorem{definition}[problem]{Definition}
%\newtheorem{thm}{Theorem}[section] 
%\newtheorem{defn}[thm]{Definition}
%\newtheorem{algorithm}{Algorithm}[section]
%\newtheorem{cor}{Corollary}
\newcommand{\BEQA}{\begin{eqnarray}}
\newcommand{\EEQA}{\end{eqnarray}}
\newcommand{\define}{\stackrel{\triangle}{=}}
\bibliographystyle{IEEEtran}
%\bibliographystyle{ieeetr}
\providecommand{\mbf}{\mathbf}
\providecommand{\pr}[1]{\ensuremath{\Pr\left(#1\right)}}
\providecommand{\qfunc}[1]{\ensuremath{Q\left(#1\right)}}
\providecommand{\sbrak}[1]{\ensuremath{{}\left[#1\right]}}
\providecommand{\lsbrak}[1]{\ensuremath{{}\left[#1\right.}}
\providecommand{\rsbrak}[1]{\ensuremath{{}\left.#1\right]}}
\providecommand{\brak}[1]{\ensuremath{\left(#1\right)}}
\providecommand{\lbrak}[1]{\ensuremath{\left(#1\right.}}
\providecommand{\rbrak}[1]{\ensuremath{\left.#1\right)}}
\providecommand{\cbrak}[1]{\ensuremath{\left\{#1\right\}}}
\providecommand{\lcbrak}[1]{\ensuremath{\left\{#1\right.}}
\providecommand{\rcbrak}[1]{\ensuremath{\left.#1\right\}}}
\theoremstyle{remark}
\newtheorem{rem}{Remark}
\newcommand{\sgn}{\mathop{\mathrm{sgn}}}
\providecommand{\abs}[1]{\ensuremath{\left\vert#1\right\vert}}
\providecommand{\res}[1]{\Res\displaylimits_{#1}} 
\providecommand{\norm}[1]{\ensuremath{\left\lVert#1\right\rVert}}
%\providecommand{\norm}[1]{\lVert#1\rVert}
\providecommand{\mtx}[1]{\mathbf{#1}}
\providecommand{\mean}[1]{\ensuremath{E\left[ #1 \right]}}
\providecommand{\fourier}{\overset{\mathcal{F}}{ \rightleftharpoons}}
%\providecommand{\hilbert}{\overset{\mathcal{H}}{ \rightleftharpoons}}
\providecommand{\system}{\overset{\mathcal{H}}{ \longleftrightarrow}}
	%\newcommand{\solution}[2]{\textbf{Solution:}{#1}}
\newcommand{\solution}{\noindent \textbf{Solution: }}
\newcommand{\cosec}{\,\text{cosec}\,}
\providecommand{\dec}[2]{\ensuremath{\overset{#1}{\underset{#2}{\gtrless}}}}
\newcommand{\myvec}[1]{\ensuremath{\begin{pmatrix}#1\end{pmatrix}}}
\newcommand{\mydet}[1]{\ensuremath{\begin{vmatrix}#1\end{vmatrix}}}
%\numberwithin{equation}{section}
\numberwithin{equation}{subsection}
%\numberwithin{problem}{section}
%\numberwithin{definition}{section}
\makeatletter
\@addtoreset{figure}{problem}
\makeatother
\let\StandardTheFigure\thefigure
\let\vec\mathbf
%\renewcommand{\thefigure}{\theproblem.\arabic{figure}}
\renewcommand{\thefigure}{\theproblem}
%\setlist[enumerate,1]{before=\renewcommand\theequation{\theenumi.\arabic{equation}}
%\counterwithin{equation}{enumi}
%\renewcommand{\theequation}{\arabic{subsection}.\arabic{equation}}
\def\putbox#1#2#3{\makebox[0in][l]{\makebox[#1][l]{}\raisebox{\baselineskip}[0in][0in]{\raisebox{#2}[0in][0in]{#3}}}}
     \def\rightbox#1{\makebox[0in][r]{#1}}
     \def\centbox#1{\makebox[0in]{#1}}
     \def\topbox#1{\raisebox{-\baselineskip}[0in][0in]{#1}}
     \def\midbox#1{\raisebox{-0.5\baselineskip}[0in][0in]{#1}}
\vspace{3cm}
\title{Assignment 11}
\author{Vishal Ashok}
\maketitle
\newpage
%\tableofcontents
\bigskip
\renewcommand{\thefigure}{\theenumi}
\renewcommand{\thetable}{\theenumi}
\begin{abstract}
This document gives us information to find transformation matrix for a given function.
\end{abstract}

%
\begin{lstlisting}
https://github.com/vishalashok98/AI5006\end{lstlisting}
%
Download latex-tikz codes from 
%
\begin{lstlisting}
https://github.com/vishalashok98/AI5006\end{lstlisting}
%
\section{Problem}
Consider the vector space $P_n$ of real polynomials in $x$ of degree less than or equal to $n$. Define $T:P_2 \rightarrow P_3$ by $(Tf)(x)=\int_{0}^{x}f(t)dt+f'(x)$.Then the matrix representation of T with respect to the bases ${1,x,x^2}$ and ${1,x,x^2,x^3}$ is

1.\begin{bmatrix}
0 & 1 & 0 & 0\\1 & 0 & \frac{1}{2} & 0\\0 & 2 & 0 & \frac{1}{3}
\end{bmatrix}
2.\begin{bmatrix}
0 & 1 & 0\\1 & 0 & 2\\0 & \frac{1}{2} & 0\\0 & 0 & \frac{1}{3}
\end{bmatrix}\newline
3.\begin{bmatrix}
0 & 1 & 0 & 0 \\1 & 0 & 2 & 0\\0 & \frac{1}{2} & 0 & \frac{1}{3}
\end{bmatrix}
4.\begin{bmatrix}
0 & 1 & 0\\1 & 0 & \frac{1}{2}\\0 & 2 & 0\\0 & 0 & \frac{1}{3}
\end{bmatrix}
    


\section{Explanation}

Transformation matrix gives the co-ordinates of images of vectors which are mapped from domain to co-domain. 








\section{Solution}

${1,x,x^2} $ is basis for $P_2$ which is domain space.

$(Tf)(x)=\int_{0}^{x}f(t)dt+f'(x)$

We should the image of basis vectors of domain space in co-domain space with respect to T.
\newline
Basis vector 

\begin{bmatrix}
            1\\x^{2} \\x^3
\end{bmatrix}

Image of basis vector in co-domain space is given by \newline
$(Tf)(x)=\int_{0}^{x} f(t)dt+f'(x)$\newline

$(Tf)(x)=\int_{0}^{x} 1+t+t^2 dt+ 1+2x$\newline
$(Tf)(x)=t+\frac{t^2}{2}+\frac{t^3}{3}\Biggr|_{0}^{x}+1+2x$\newline
$(Tf)(x)=3x+\frac{x^3}{3}+\frac{x^2}{2}+1$

So coordinates of image of base vector is given by 


Base vector 1:$f(x)=1$
\newline
$f'(x)=0$
$T(1)=\int_{0}^{x} 1dt+0=t\Biggr|_{0}^{x}$
\newline
It can be written as linear combination of basis vectors of co-domain space.
\newline
$x=0+1.x+0.x^2+0.x^3$
\newline
So coordinates of $T(1)$ with respect to co-domain basis are\begin{bmatrix}
1\\3\\\frac{1}{2}\\\frac{1}{3}
\end{bmatrix}
Co ordinates of basis vectors in domain space are \begin{bmatrix}
1\\1\\1
\end{bmatrix}

Among the given options matrix which maps coordinates of basis vector from domain space
to co-domain space is

$[T]$=\begin{bmatrix}
0 & 1 & 0\\1 & 0 & 2\\0 & \frac{1}{2} & 0\\0 & 0 & \frac{1}{3}
\end{bmatrix}










\end{document}
